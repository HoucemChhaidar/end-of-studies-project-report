\chapter*{General Introduction}
\addcontentsline{toc}{chapter}{General Introduction}
\markboth{General Introduction}{General Introduction}
\adjustmtc

In recent years, the rapid advancement of information and communication technologies has brought about a profound transformation across nearly every sector of society. Among the most impacted domains is the financial and retail industry, where digitalization has reshaped how payments are processed, managed, and secured. As consumers increasingly seek convenience, speed, and security, mobile payment systems have emerged as a powerful alternative to traditional cash-based transactions.

This shift is particularly evident in Tunisia, where the adoption of mobile technologies is accelerating in sectors such as food service and retail. Yet, despite the promise of digital payments, several technical and operational challenges persist—ranging from secure transaction validation and fraud detection to user authentication and real-time system responsiveness.

In response to these evolving needs, this end-of-studies project proposes the design and implementation of a mobile payment system in collaboration with ASM, a Tunisian company specializing in Point of Sale (POS) solutions for restaurants and stores. The project aims to deliver a fully integrated platform that connects a Flutter-based mobile application with ASM's POS infrastructure through a Spring Boot backend, allowing seamless barcode-based payments, corporate credit management, and vendor transaction processing.

Beyond its functional scope, the system addresses key challenges such as short-lived token generation, biometric authentication, offline transaction handling, and multilingual user experience (supporting Arabic, French, and English). Furthermore, the solution supports corporate-level features including bulk credit distribution, spending limits, and real-time monitoring—making it suitable for enterprise-wide deployment.

This thesis is organized into several chapters to provide a structured presentation of the project. The first chapter introduces the host organization and outlines the project objectives, followed by an analytical study of existing solutions. The second chapter elaborates on the proposed system architecture, detailing both functional and non-functional requirements, along with the rationale behind the technological choices made. The third and fourth chapters document the development process, divided into iterative sprints, highlighting implementation strategies, challenges encountered, and solutions applied.

The report concludes with a critical evaluation of the completed work, identifying potential areas for improvement and outlining perspectives for future enhancements.